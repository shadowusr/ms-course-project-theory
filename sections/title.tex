\newgeometry{left=15mm, right=15mm, top=20mm, bottom=20mm}
\begin{center}
\small
МИНИСТЕРСТВО НАУКИ И ВЫСШЕГО ОБРАЗОВАНИЯ РОССИЙСКОЙ ФЕДЕРАЦИИ\\
{\normalsize Федеральное государственное автономное образовательное учреждение высшего образования}\\
«КРЫМСКИЙ ФЕДЕРАЛЬНЫЙ УНИВЕРСИТЕТ имени В. И. ВЕРНАДСКОГО»\\
(ФГАОУ ВО «КФУ им. В. И. Вернадского»)\\
\normalsize
Таврическая академия (структурное подразделение)\\
Факультет математики и информатики\\
Кафедра информатики\\
\vspace*{2\baselineskip}
Марков Николай Дмитриевич\\
\vspace*{1\baselineskip}
\textbf{РАЗРАБОТКА ОБРАЗОВАТЕЛЬНОГО WEB--СЕРВИСА С ИСПОЛЬЗОВАНИЕМ АЛГОРИТМОВ МАШИННОГО ОБУЧЕНИЯ}\\
\vspace*{2\baselineskip}
Курсовой проект\\
\vspace*{1\baselineskip}
\end{center}

%\normalsize{ \hspace{28pt} Допущено к защите в ГЭК  27.05.2015} 
 
\begin{tabular}{m{12em} m{10em}}
	%Зав.кафедрой & \underline{\hspace{3cm}} &  д.физ.-мат.н.,  проф. & Е.М. Семёнов \\\\
	%Обучающийся & \underline{\hspace{3cm}} & &С.М. Петров \\\\
	%Руководитель & \underline{\hspace{3cm}}& д.физ.-мат.н., проф.&  Т.Я. Азизов \\\\
	обучающегося & \underline{1 курса}\\
	направления подготовки &\underline{01.04.02 Прикладная математика и информатика}\\
	форма обучения &\underline{очная}\\
\end{tabular}\\

\vspace*{4\baselineskip}

\begin{adjustwidth}{2em}{0pt}
\noindent
Научный руководитель\\
доцент кафедры информатики\\
кандидат физико-математических наук
\null\hfill А. С. Анафиев
\end{adjustwidth}

\vspace*{1\baselineskip}

%\begin{adjustwidth}{2em}{0pt}
%\noindent
%Оценка руководителя:  \underline{\hspace{7em}} \null\hfill \underline{\hspace{7em}}
%\end{adjustwidth}

\vfill

\begin{center} Симферополь --- 2022 \end{center}
\thispagestyle{empty} % выключаем отображение номера для этой страницы
\restoregeometry